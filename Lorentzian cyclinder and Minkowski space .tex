% filepath: /Users/davidengland/Documents/GitHub/clifford/Articles/LorentzianCylinder_Minkowski_Conformal.tex
\documentclass[11pt]{article}

\usepackage[a4paper,margin=1in]{geometry}
\usepackage{amsmath,amssymb,amsthm,mathtools}
\usepackage{bm}
\usepackage{hyperref}
\usepackage[nameinlink]{cleveref}
\usepackage{tikz}
\usepackage{tikz-cd}
\usepackage{physics}
\usepackage{microtype}
\usepackage{enumitem}

\hypersetup{
  colorlinks = true,
  linkcolor = blue!60!black,
  citecolor = blue!60!black,
  urlcolor  = blue!60!black
}

\title{Lorentzian Cylinder vs.\ Minkowski Space in 1+1 Dimensions:\\
Global Conformal Non--Equivalence and Shared Symmetry Structures}
\author{%
  % ...add author(s)...
}
\date{\today}

% Theorems
\newtheorem{claim}{Claim}
\newtheorem{prop}{Proposition}
\newtheorem{thm}{Theorem}
\theoremstyle{remark}
\newtheorem{remark}{Remark}
\newtheorem{example}{Example}

% Notation
\newcommand{\R}{\mathbb{R}}
\newcommand{\Sone}{\mathbb{S}^{1}}
\newcommand{\M}{\mathbb{R}^{1,1}}
\newcommand{\Ecyl}{\R\times \Sone}
\newcommand{\SO}{\mathrm{SO}}
\newcommand{\SL}{\mathrm{SL}}
\newcommand{\PSL}{\mathrm{PSL}}
\newcommand{\Diff}{\mathrm{Diff}}
\newcommand{\ul}{\underline}
\newcommand{\ol}{\overline}

% Null coordinates
\newcommand{\uu}{u}
\newcommand{\vv}{v}
\newcommand{\UU}{U}
\newcommand{\VV}{V}
\newcommand{\TT}{T}
\newcommand{\XX}{X}
\newcommand{\ttt}{t}
\newcommand{\thth}{\theta}

\begin{document}
\maketitle

\begin{abstract}
We give a concise, self-contained treatment of the conformal geometry of two-dimensional Minkowski space $\M$ and the Lorentzian Einstein cylinder $\Ecyl$ with metric $-d\ttt^2 + d\thth^2$. We prove their \emph{global} conformal non--equivalence via topological obstructions and Penrose compactification, while emphasizing their \emph{shared} conformal symmetry structures: locally $\Diff$ of the two null lines, globally the finite conformal group $\SO^{+}(2,2)\cong (\PSL(2,\R)\times \PSL(2,\R))/\mathbb{Z}_2$. We include explicit conformal embeddings, the role of periodicity constraints on the cylinder, generators in null coordinates, and multiple complementary viewpoints (covering spaces, complex analysis, and CFT). The target audience is theoretical physicists; the presentation is intended as a bridge between GR, QFT/CFT, and 2D Lorentzian geometry.
\end{abstract}

\paragraph{Keywords.} Einstein cylinder, conformal compactification, $\SO(2,2)$, Witt/Virasoro algebra, Penrose diagram, 1+1 D Lorentzian geometry.

\section{Introduction and summary}
It is standard folklore (and textbook material) that $1+1$ Minkowski space $(\M, -d\TT^2 + d\XX^2)$ admits a global conformal compactification into an open diamond of the Einstein cylinder $(\Ecyl, -d\ttt^2 + d\thth^2)$, and that both admit large conformal symmetry groups. However, one must distinguish:
\begin{itemize}[leftmargin=*,nosep]
\item \emph{Local} conformal transformations (independent reparametrizations of null coordinates), with Lie algebra the Witt algebra in each null sector.
\item \emph{Global} finite conformal transformations that extend to the conformal compactification, forming $\SO^{+}(2,2)\cong (\PSL(2,\R)\times \PSL(2,\R))/\mathbb{Z}_2$.
\end{itemize}
Despite sharing the same finite global conformal group, the two spacetimes are \emph{not} globally conformally equivalent, due to topology ($\pi_1$) and global causal structure. We expand these points with proofs, explicit maps, and several perspectives useful in GR and CFT.

\section{Local conformal geometry in null coordinates}
On $\M$, set null coordinates $\uu=\TT+\XX$, $\vv=\TT-\XX$ so that
\begin{equation}
  -d\TT^2 + d\XX^2 = -\, d\uu\, d\vv.
\end{equation}
On the cylinder $\Ecyl$ with coordinates $(\ttt,\thth)\in\R\times \Sone$, set $\uu=\ttt+\thth$, $\vv=\ttt-\thth$, so that
\begin{equation}
  -d\ttt^2 + d\thth^2 = -\, d\uu\, d\vv,
\end{equation}
subject to the global identification
\begin{equation}
  (\uu,\vv) \sim (\uu+2\pi, \vv+2\pi).
\end{equation}
Locally, any conformal diffeomorphism has the form
\begin{equation}
  (\uu,\vv) \longmapsto \big(\phi(\uu),\, \psi(\vv)\big), \qquad \phi',\psi'>0,
\end{equation}
and the metric transforms by the conformal factor
\begin{equation}
  -d\uu\, d\vv \ \longmapsto\ -\phi'(\uu)\,\psi'(\vv)\, d\uu\, d\vv
  \ =\ \Omega^{2}(\uu,\vv) \, (-d\uu\, d\vv), \qquad \Omega^{2}=\phi'(\uu)\psi'(\vv).
\end{equation}
Thus the local conformal Lie algebra is generated by vector fields
\begin{equation}
  X=\xi(\uu)\,\partial_{\uu} \ +\ \eta(\vv)\,\partial_{\vv},
\end{equation}
i.e.\ two commuting copies of the Witt algebra.

\begin{remark}[Global constraint on the cylinder]\label{rem:periodicity}
On $\Ecyl$, $(\uu,\vv)\sim (\uu+2\pi,\vv+2\pi)$. A \emph{global} conformal diffeomorphism must respect this identification:
\begin{equation}
  \phi(\uu+2\pi)=\phi(\uu)+2\pi,\qquad \psi(\vv+2\pi)=\psi(\vv)+2\pi,
\end{equation}
so that $\phi,\psi$ descend to $\Diff(\Sone)$ (orientation-preserving component), not arbitrary $\Diff(\R)$.
\end{remark}

\section{Global non--equivalence}\label{sec:nonequiv}
\begin{claim}[Topological obstruction]\label{cl:pi1}
$(\M,-d\TT^2+d\XX^2)$ and $(\Ecyl,-d\ttt^2+d\thth^2)$ are not globally conformally equivalent.
\end{claim}
\begin{proof}
Any conformal equivalence is, in particular, a diffeomorphism. But
$\pi_1(\M)=0$ while $\pi_1(\Ecyl)\cong \mathbb{Z}$. Since $\pi_1$ is a diffeomorphism invariant, no global conformal diffeomorphism exists.
\end{proof}

\begin{prop}[Penrose compactification viewpoint]\label{prop:penrose}
There exists a conformal embedding of $\M$ into an \emph{open diamond} in $\Ecyl$, but not onto all of $\Ecyl$.
\end{prop}
\begin{proof}
Define on $\M$ the compactifying coordinates $\UU=\arctan \uu$, $\VV=\arctan \vv$ with $\UU,\VV\in(-\tfrac{\pi}{2}, \tfrac{\pi}{2})$. Then set
\begin{equation}
  \ttt = \frac{\UU+\VV}{2}, \qquad \thth = \frac{\UU-\VV}{2}.
\end{equation}
A direct computation shows
\begin{equation}
  -d\uu\, d\vv = \frac{1}{\cos^{2}\UU\,\cos^{2}\VV}\, (-d\UU\, d\VV) \ =\ \Omega^{2}\, (-d\ttt^{2}+d\thth^{2}),
\end{equation}
with a smooth, positive $\Omega$ on the image. The range is the open set
$\{(\ttt,\thth):\, |\ttt\pm \thth|<\tfrac{\pi}{2}\}$, i.e.\ a proper subset (a diamond) of $\Ecyl$.
\end{proof}

\begin{remark}[Universal covers and spatial compactness]
A global conformal diffeomorphism would induce an isomorphism of universal covers; but the spatial factor of the cylinder is compact ($\Sone$), while $\M$ has non-compact space slices, preventing such an identification.
\end{remark}

\section{Finite global conformal group}\label{sec:finite}
The global conformal group that extends to the (double) conformal compactification in $1+1$ dimensions is the identity-connected component
\begin{equation}
  \SO^{+}(2,2)\ \simeq\ \frac{\PSL(2,\R)\times \PSL(2,\R)}{\mathbb{Z}_{2}}.
\end{equation}
Its Lie algebra is
$\mathfrak{so}(2,2)\cong \mathfrak{sl}(2,\R)\oplus \mathfrak{sl}(2,\R)$, generated in each null sector by three vector fields. In coordinates on $\R$ one uses $\{1,\uu,\uu^{2}\}$ (and analogously in $\vv$); on $\Sone$ one uses the Möbius generators (equivalently, trigonometric combinations $\{1,\sin \sigma,\cos \sigma\}$ after appropriate linear combinations).

\begin{remark}[Components, orientations]
$\SO(2,2)$ has several connected components. In physical applications one often restricts to orientation- and time-orientation--preserving components ($\SO^{+}(2,2)$).
\end{remark}

\section{Local symmetries: Witt vs.\ Virasoro}\label{sec:witt-vir}
The local conformal algebra is two copies of the Witt algebra of vector fields on the null line/circle. In quantum 2D CFT, the relevant algebra is the \emph{Virasoro} algebra, the unique (projective) central extension of the Witt algebra, with central charges $(c,\bar c)$. This refinement does not affect the classical global non--equivalence, but it governs operator algebras and state spaces on the cylinder vs.\ plane.

\section{Alternative perspectives and useful constructions}
\paragraph{(i) Complex analysis.} In Euclidean signature, one uses holomorphic maps $z\mapsto f(z)$ and $\bar z \mapsto \overline{f(z)}$. In Lorentzian signature, null coordinates $(\uu,\vv)$ play an analogous role. On the cylinder, use $e^{i(\ttt\pm\thth)}$ as lightlike coordinates for global analyses.

\paragraph{(ii) Covering spaces.} The spatial circle implies $(\uu,\vv)\sim (\uu+2\pi,\vv+2\pi)$. The universal cover of the cylinder is $\R^2$ with the same local metric $-d\uu\, d\vv$, but global maps must commute with the deck transformation $(\uu,\vv)\mapsto (\uu+2\pi,\vv+2\pi)$, which yields the periodicity constraints in \Cref{rem:periodicity}.

\paragraph{(iii) QFT/CFT data.} On the plane, global conformal symmetry is $\PSL(2,\R)\times\PSL(2,\R)$; on the cylinder, the Hamiltonian is related to $L_0+\bar L_0$ (plus Casimir/normal-ordering terms), while spatial translations correspond to $L_0-\bar L_0$. Two-point functions transform with the conformal factor; e.g.\ for a primary of weights $(h,\bar h)$,
\begin{equation}
  \langle \mathcal{O}(\uu,\vv)\mathcal{O}(0,0)\rangle \propto (\uu-i0)^{-2h}(\vv-i0)^{-2\bar h}
\end{equation}
on the plane, and pick up periodic images on the cylinder.

\paragraph{(iv) Boundary/corner structure in compactification.} The embedding of $\M$ into the cylinder maps $\scri^\pm$ (null infinities) to segments on the boundary of the diamond, with $i^{0}, i^{\pm}$ mapped to corners. This is the $1+1$ analogue of standard 3+1 Penrose diagrams.

\section{Worked examples}
\begin{example}[Explicit compactification map]\label{ex:penrose-map}
Let $\UU=\arctan \uu$, $\VV=\arctan \vv$, $\ttt=(\UU+\VV)/2$, $\thth=(\UU-\VV)/2$. Then
\begin{equation}
  -d\uu\, d\vv = \frac{1}{\cos^{2}\UU\,\cos^{2}\VV}\, (-d\UU\, d\VV) = \Omega^{2}(\ttt,\thth)\, (-d\ttt^{2}+d\thth^{2}),
\end{equation}
with $\Omega^{-2} = \cos^{2}(\ttt+\thth)\,\cos^{2}(\ttt-\thth)$. The image is the diamond $|\ttt\pm \thth|<\tfrac{\pi}{2}$ in the cylinder.
\end{example}

\begin{example}[Global maps on the cylinder]
Let $\phi\in \Diff(\Sone)$ be the lift of a circle diffeomorphism to $\R$ obeying $\phi(\uu+2\pi)=\phi(\uu)+2\pi$. Then
\begin{equation}
  (\uu,\vv)\mapsto \big(\phi(\uu),\psi(\vv)\big), \qquad \Omega^{2}=\phi'(\uu)\psi'(\vv),
\end{equation}
defines a global conformal diffeomorphism of the cylinder (orientation-preserving component).
\end{example}

\section{Conclusions}
We have:
\begin{itemize}[leftmargin=*,nosep]
\item Established the \emph{global} conformal non--equivalence $\M \not\cong \Ecyl$ (as conformal manifolds) via topology and compactification.
\item Exhibited their shared \emph{local} conformal structure (independent null reparametrizations) and the shared \emph{finite} global conformal group $\SO^{+}(2,2)$.
\item Clarified global constraints (periodicity), components (orientation/time-orientation), and connected these structures to CFT (Virasoro) and GR (Penrose diagrams).
\end{itemize}
These $1+1$ structures provide a clean laboratory for studying the interplay of global geometry, symmetry, and field-theoretic structures.

\section*{Acknowledgements}
% ...optional...

\begin{thebibliography}{9}

\bibitem{HawkingEllis}
S.~W.~Hawking and G.~F.~R.~Ellis,
\emph{The Large Scale Structure of Space-Time}, Cambridge University Press (1973).

\bibitem{Wald}
R.~M.~Wald, \emph{General Relativity}, University of Chicago Press (1984), App.~C.

\bibitem{DiFrancesco}
P.~Di~Francesco, P.~Mathieu, D.~S\'en\'echal, \emph{Conformal Field Theory}, Springer (1997), Ch.~4.

\bibitem{Frances}
C.~Frances, \emph{Conformal Geometry and Lorentzian Manifolds}, lecture notes.

\bibitem{Penrose}
R.~Penrose, \emph{Conformal treatment of infinity}, in \emph{Relativity, Groups and Topology}, Gordon and Breach (1964).

\end{thebibliography}

\end{document}